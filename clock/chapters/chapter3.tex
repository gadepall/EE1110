%\subsection{Driving the Segments}
%
\begin{problem}
Execute the following code.  What do you observe?
\end{problem}	
%
\lstinputlisting[language=C]{./codes/clock_single.c}
%
%
\begin{problem}
Connect one more display to the breadboard and complete the hardware setup.  A parallel connection from the D2-D8 pins should be made to the $a-g$ pins of this display.
\end{problem}	
%
%
\begin{problem}
Connect the D10 pin of the arduino to the free COM pin of the second display and execute the following code.  What do you observe?
\end{problem}	

\lstinputlisting[language=C]{./codes/clock_double.c}
%
%
\begin{problem}
Connect 4 more displays and build a 24 hour digital clock.
\end{problem}

%\begin{problem}
%Connect the A-D pins of the 7447 IC to the pins D2-D5 of the Arduino.
%\end{problem}	
%\begin{problem}
%Type the following code and execute. What do you observe? Explain  through Fig. \ref{fig:Atmega168PinMap2}
%\lstinputlisting[language=C]{./codes/decoder}
%%\input{ard_decoder_drive}
%\end{problem}
%%
%\begin{problem}
%Now generate the numbers 0-9 by modifying the above program.
%\end{problem}
%%
%\begin{problem}
%Type the following code and execute. Comment
%\end{problem}
%%
%\lstinputlisting[language=C]{./codes/loop_counter}
%%
%\begin{figure}[!h]
%\begin{center}
%\includegraphics[width=\columnwidth]{./figs/Atmega168PinMap2}
%\end{center}
%\caption{}
%\label{fig:Atmega168PinMap2}
%\end{figure}
%%\newpage

%\section{Combinational Logic}
%%
%Test all the following .using the 7447 IC
%\subsection{Counting Decoder}
%%
%\begin{problem}
%Write an assembly program for the AND operation.  
%\end{problem}
%\solution
%\lstinputlisting[language=C]{./codes/logic_and}
%%
%\begin{problem}
%Write an assembly program for the OR operation.  
%\end{problem}
%\solution
%\lstinputlisting[language=C]{./codes/logic_or}
%%
%\begin{problem}
%Write an assembly program for the XOR operation.  
%\end{problem}
%\solution
%\lstinputlisting[language=C]{./codes/logic_xor}
%%
%\begin{problem}
%Write an assembly program to complement a bit using the XOR operation.
%\end{problem}
%%
%%
%\begin{problem}
%Write an assembly routine to complement a bit. Call this routine by using RCALL.
%\end{problem}

%%\solution
%%\lstinputlisting[language=C]{./codes/complement}

%\begin{problem}
	%\label{counter_dec}
	%In the  truth table in Table \ref{table:counter_decoder},  $W,X,Y,Z$ are the inputs
%and $A,B,C,D$ are the outputs. This table represents the system that increments the numbers 0-8 by 1 and resets the number 9 to 0
%%
%Note that  $D = 1$ for the inputs $0111$ and $1000$.  Using {\em boolean} logic,
%%
%\begin{equation}
%\label{bool_logic}
%D = WXYZ^{\prime} + W^{\prime}X^{\prime}Y^{\prime}Z
%\end{equation}
%%
%Note that $0111$ results in the expression $WXYZ^{\prime}$ and $1000$ yields $W^{\prime}X^{\prime }Y^{\prime}Z$. 
%Write the boolean logic functions for $A,B,C$ in terms of $W,X,Y,Z$.
%\end{problem}
%%
%\solution The desired equations are
%%
%\begin{align}
%A &= W^{\prime}
%\\
%B &= WX^{\prime} Z^{\prime} + W^{\prime}X
%\\
%C &= WXY^{\prime}+X^{\prime}Y + W^{\prime}Y
%\\
%D &= WXY + W^{\prime}Z
%\end{align}
%%
%%
%\begin{problem}
%Write an assembly program for implementing Table \ref{table:counter_decoder} and verify if your logic is correct by observing the output on the seven segment display.
%\end{problem}
%%
%%\begin{problem}
%%Use the following code for LED blinking. You will have to connect pin 13 to the LED on the seven segment display.
%%\end{problem}
%%\lstinputlisting[language=C]{./codes/blink}
%%%
%%\begin{problem}
%%Find the exact blinking delay for the above.  Verify your result through an oscilloscope. 
%%\end{problem}
%%%
%%\begin{problem}
%%Modify the above code to get the blinking delay to 1 second.  Verify your results.
%%\end{problem}
%%%
%%\begin{problem}
%%Implement a decade counter.
%%\end{problem}
%\input{./figs/counter_decoder}
%%The $\&\&$ operand is used for the boolean AND (multiplication) operation, the $||$ operand is used for the OR (addition) operation and the ! operand is used for the NOT ($^{'}$) operation in Arduino code.  For example, the expression for \eqref{bool_logic} in Arudino is
%%\begin{verbatim}
%%D = (W&&X&&Y&&!Z)||(!W&&!X&&!Y&&Z);
%%\end{verbatim}
%\subsection{Display Decoder}
%%
%\begin{problem}
%Now write the truth table for the seven segment display decoder (IC 7447).  The inputs will be $A,B,C,D$ and the outputs will be $a,b,c,d,e,f,g$.
%\end{problem}
%%
%\begin{problem}
%\label{seven_seg_disp_logic}
%Obtain the logic functions for outputs $a,b,c,d,e,f,g$ in terms of the inputs $A,B,C,D$.
%\end{problem}
%\begin{problem}
%Disconnect the arduino from IC 7447 and connect the pins D2-D8 in the Arduino directly to the seven segment display.
%\end{problem}
%\begin{problem}
%Write a new program to implement the logic in Problem \ref{seven_seg_disp_logic} and observe the output in the display.  You have designed the logic for IC 7447!
%\end{problem}
%\begin{problem}
%Now include your counting decoder program in the  display decoder program
%and see if the display shows the consecutive number.
%\end{problem}
%%A decade counter counts the numbers from 0-9 and then resets to 0.
%%\begin{problem}
%%Suitably modify the above program to obtain a decade counter.
%%\end{problem}





%%\begin{problem}
%%Generate the boolean functions for the segments $a-f$ using the table in Problem \ref{bcd_ss}.  For example, the function for $a$ is obtained from the table as
%%\begin{equation}
%%a=\bar{D}\bar{C}\bar{B}A+\bar{D}C\bar{B}\bar{A}
%%\label{boolean}
%%\end{equation}
%%\end{problem}
%%%
%%\begin{problem}
	%%\label{counter_dec}
%%Write functions for $A,B,C,D$ in Arduino using the following table and verify using the Arduino driven display.
		%%\input{counter_decoder}
%%\end{problem}
%%\begin{problem}
	%%Write a module for decimal to binary conversion
	%%according to the example given below
	%%\input{conversion}
	%%%
	%%$N \% 2$ gives the remainder and $N/2$ gives the quotient
%%	and use it in the above code so that decimal values are given as input in the program and observed as output in the display. Note that the following code
%%	\begin{verbatim}
%%	a % b
%%	\end{verbatim}
%%	can be used to obtain the remainder when a is divided by b and
%%	\begin{verbatim}
%%	a/b
%%	\end{verbatim}
%%	gives the quotient.
%%\end{problem}
